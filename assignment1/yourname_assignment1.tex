\documentclass[11pt]{article}

\RequirePackage[l2tabu, orthodox]{nag}
%\documentclass{article}

\usepackage[left=1.in, right=1.in, top=1.25in, bottom=1.25in]{geometry}

% FONTS
%\usepackage[T1]{fontenc}

% Replace default Latin Modern typewriter with its proportional counterpart
% http://www.tug.dk/FontCatalogue/lmoderntypewriterprop/
%\renewcommand*\ttdefault{lmvtt}


%%% OPTION 1 - Fourier Math + New Century Schoolbook + ParaType Sans

% % Import Fourier Math (this imposes its own New Century Schoolbook type)
% % http://www.ctan.org/tex-archive/fonts/fouriernc/
%\usepackage{fouriernc}
%\usepackage{amsmath}
% % Replace with TeX Gyre Schola version of New Century Schoolbook (must scale!)
% % http://www.tug.dk/FontCatalogue/tgschola/
%\usepackage[scale=0.92]{tgschola}
%\usepackage[scaled=0.88]{PTSans}

%% OPTION 2 - MathDesign Math + Bitstream Charter + ParaType Sans

% Import MathDesign (this brings along Bitstream Charter)
% http://www.ctan.org/tex-archive/fonts/mathdesign/
\usepackage[bitstream-charter]{mathdesign}
\usepackage{amsmath}
\usepackage[scaled=0.92]{PTSans}


% %%% OPTION 3 - MTPRO 2 Math + Termes Times + ParaType Sans

% \usepackage{tgtermes}
% \usepackage{amsmath}
% \usepackage[subscriptcorrection,
%             amssymbols,
%             mtpbb,
%             mtpcal,
%             nofontinfo  % suppresses all warnings
%            ]{mtpro2}
% \usepackage{scalefnt,letltxmacro}
% \LetLtxMacro{\oldtextsc}{\textsc}
% \renewcommand{\textsc}[1]{\oldtextsc{\scalefont{1.10}#1}}
% \usepackage[scaled=0.92]{PTSans}

% Use default fonts here
% \usepackage{amsmath}
% \usepackage{amssymb}

\usepackage{titling}

% COLOR
\usepackage[table,usenames,dvipsnames]{xcolor}
\definecolor{shadecolor}{gray}{0.9}

% SPACING and TEXT
\usepackage[final,expansion=alltext]{microtype}
\usepackage[english]{babel}
\usepackage[parfill]{parskip}
\usepackage{afterpage}
\usepackage{framed}
\usepackage{verbatim}
\usepackage{setspace}

%redefine the leftbar environment to accept a width and coloring options
\renewenvironment{leftbar}[1][\hsize]
{%
  \def\FrameCommand
  {%
    {\color{Gray}\vrule width 3pt}%
    \hspace{10pt}%
    %\hspace{0pt}\fboxsep=\FrameSep\colorbox{black!10}%
  }%
  \MakeFramed{\hsize#1\advance\hsize-\width\FrameRestore}%
}%
{\endMakeFramed}

% define a paragraph header function
\DeclareRobustCommand{\parhead}[1]{\textbf{#1}~}

% EDITING
% line numbering in left margin
\usepackage{lineno}
\renewcommand\linenumberfont{\normalfont
                             \footnotesize
                             \sffamily
                             \color{SkyBlue}}
% ragged paragraphs in right margin
\usepackage{ragged2e}
\DeclareRobustCommand{\sidenote}[1]{\marginpar{
                                    \RaggedRight
                                    \textcolor{Plum}{\textsf{#1}}}}
% paragraph counter in right margin
\newcommand{\parnum}{\bfseries\P\arabic{parcount}}
\newcounter{parcount}
\newcommand\p{%
    \stepcounter{parcount}%
    \leavevmode\marginpar[\hfill\parnum]{\parnum}%
}
% paragraph helper
%\DeclareRobustCommand{\PP}{\textcolor{Plum}{\P} }

% \usepackage[bottom]{footmisc}
\usepackage[symbol]{footmisc}
\renewcommand{\thefootnote}{\arabic{footnote}}

% COUNTERS
\usepackage[inline]{enumitem}
\renewcommand{\labelenumi}{\color{black!67}{\arabic{enumi}.}}
\renewcommand{\labelenumii}{{\color{black!67}(\alph{enumii})}}
\renewcommand{\labelitemi}{{\color{black!67}\textbullet}}

% FIGURES
\usepackage{graphicx}
\usepackage[labelfont={it, small}, font=small]{caption}
\usepackage[format=hang]{subcaption}
% \usepackage{ccaption}

% APPENDIX FIGURES
\usepackage{chngcntr}

% TABLES
\usepackage{booktabs}
\usepackage{longtable}
\usepackage{hhline}

% ALGORITHMS
\usepackage[algoruled]{algorithm2e}
\usepackage{listings}
\usepackage{fancyvrb}
\fvset{fontsize=\normalsize}

% THEOREMS
\usepackage{amsthm}
\newtheorem{proposition}{Proposition}
\newtheorem{lemma}{Lemma}

% BIBLIOGRAPHY
\usepackage[numbers]{natbib}

% HYPERREF
\usepackage[colorlinks,linktoc=all]{hyperref}
\usepackage[all]{hypcap}
\hypersetup{citecolor=MidnightBlue}
\hypersetup{linkcolor=black}
\hypersetup{urlcolor=MidnightBlue}

% CLEVEREF must come after HYPERREF
\usepackage[nameinlink]{cleveref}

% ACRONYMS
\usepackage[acronym,smallcaps,nowarn]{glossaries}
% \makeglossaries

% COLOR DEFINITIONS
\newcommand{\red}[1]{\textcolor{BrickRed}{#1}}
\newcommand{\orange}[1]{\textcolor{BurntOrange}{#1}}
\newcommand{\green}[1]{\textcolor{OliveGreen}{#1}}
\newcommand{\blue}[1]{\textcolor{MidnightBlue}{#1}}
\newcommand{\gray}[1]{\textcolor{black!60}{#1}}

% LISTINGS DEFINTIONS
\lstdefinestyle{mystyle}{
    commentstyle=\color{OliveGreen},
    keywordstyle=\color{BurntOrange},
    numberstyle=\tiny\color{black!60},
    stringstyle=\color{MidnightBlue},
    basicstyle=\ttfamily,
    breakatwhitespace=false,
    breaklines=true,
    captionpos=b,
    keepspaces=true,
    numbers=left,
    numbersep=5pt,
    showspaces=false,
    showstringspaces=false,
    showtabs=false,
    tabsize=2
}
\lstset{style=mystyle}

\usepackage[colorinlistoftodos,
            prependcaption,
            textsize=small,
            backgroundcolor=yellow,
            linecolor=lightgray,
            bordercolor=lightgray]{todonotes}

% Define an environment for solutions
\newenvironment{solution}
    {
    \color{MidnightBlue}
    }
    { 
    }
%--------------------------------------------------
% !TEX root = template.tex

% \DeclareRobustCommand{\mb}[1]{\ensuremath{\boldsymbol{\mathbf{#1}}}}
\DeclareRobustCommand{\mb}[1]{\boldsymbol{#1}}

% \newcommand{\KL}[2]{\ensuremath{\textrm{KL}\PARENS{#1\;\|\;#2}}}
\DeclareRobustCommand{\KL}[2]{\ensuremath{\textrm{KL}\left(#1\;\|\;#2\right)}}

\DeclareMathOperator*{\argmax}{arg\,max}
\DeclareMathOperator*{\argmin}{arg\,min}

\renewcommand{\mid}{~\vert~}
\newcommand{\given}{\,|\,}
\newcommand{\iid}[1]{\stackrel{\text{iid}}{#1}}

\newcommand{\mba}{\mb{a}}
\newcommand{\mbb}{\mb{b}}
\newcommand{\mbc}{\mb{c}}
\newcommand{\mbd}{\mb{d}}
\newcommand{\mbe}{\mb{e}}
% \newcommand{\mbf}{\mb{f}}
\newcommand{\mbg}{\mb{g}}
\newcommand{\mbh}{\mb{h}}
\newcommand{\mbi}{\mb{i}}
\newcommand{\mbj}{\mb{j}}
\newcommand{\mbk}{\mb{k}}
\newcommand{\mbl}{\mb{l}}
\newcommand{\mbm}{\mb{m}}
\newcommand{\mbn}{\mb{n}}
\newcommand{\mbo}{\mb{o}}
\newcommand{\mbp}{\mb{p}}
\newcommand{\mbq}{\mb{q}}
\newcommand{\mbr}{\mb{r}}
\newcommand{\mbs}{\mb{s}}
\newcommand{\mbt}{\mb{t}}
\newcommand{\mbu}{\mb{u}}
\newcommand{\mbv}{\mb{v}}
\newcommand{\mbw}{\mb{w}}
\newcommand{\mbx}{\mb{x}}
\newcommand{\mby}{\mb{y}}
\newcommand{\mbz}{\mb{z}}

\newcommand{\mbA}{\mb{A}}
\newcommand{\mbB}{\mb{B}}
\newcommand{\mbC}{\mb{C}}
\newcommand{\mbD}{\mb{D}}
\newcommand{\mbE}{\mb{E}}
\newcommand{\mbF}{\mb{F}}
\newcommand{\mbG}{\mb{G}}
\newcommand{\mbH}{\mb{H}}
\newcommand{\mbI}{\mb{I}}
\newcommand{\mbJ}{\mb{J}}
\newcommand{\mbK}{\mb{K}}
\newcommand{\mbL}{\mb{L}}
\newcommand{\mbM}{\mb{M}}
\newcommand{\mbN}{\mb{N}}
\newcommand{\mbO}{\mb{O}}
\newcommand{\mbP}{\mb{P}}
\newcommand{\mbQ}{\mb{Q}}
\newcommand{\mbR}{\mb{R}}
\newcommand{\mbS}{\mb{S}}
\newcommand{\mbT}{\mb{T}}
\newcommand{\mbU}{\mb{U}}
\newcommand{\mbV}{\mb{V}}
\newcommand{\mbW}{\mb{W}}
\newcommand{\mbX}{\mb{X}}
\newcommand{\mbY}{\mb{Y}}
\newcommand{\mbZ}{\mb{Z}}

\newcommand{\mbalpha}{\mb{\alpha}}
\newcommand{\mbbeta}{\mb{\beta}}
\newcommand{\mbdelta}{\mb{\delta}}
\newcommand{\mbepsilon}{\mb{\epsilon}}
\newcommand{\mbchi}{\mb{\chi}}
\newcommand{\mbeta}{\mb{\eta}}
\newcommand{\mbgamma}{\mb{\gamma}}
\newcommand{\mbiota}{\mb{\iota}}
\newcommand{\mbkappa}{\mb{\kappa}}
\newcommand{\mblambda}{\mb{\lambda}}
\newcommand{\mbmu}{\mb{\mu}}
\newcommand{\mbnu}{\mb{\nu}}
\newcommand{\mbomega}{\mb{\omega}}
\newcommand{\mbphi}{\mb{\phi}}
\newcommand{\mbpi}{\mb{\pi}}
\newcommand{\mbpsi}{\mb{\psi}}
\newcommand{\mbrho}{\mb{\rho}}
\newcommand{\mbsigma}{\mb{\sigma}}
\newcommand{\mbtau}{\mb{\tau}}
\newcommand{\mbtheta}{\mb{\theta}}
\newcommand{\mbupsilon}{\mb{\upsilon}}
\newcommand{\mbvarepsilon}{\mb{\varepsilon}}
\newcommand{\mbvarphi}{\mb{\varphi}}
\newcommand{\mbvartheta}{\mb{\vartheta}}
\newcommand{\mbvarrho}{\mb{\varrho}}
\newcommand{\mbxi}{\mb{\xi}}
\newcommand{\mbzeta}{\mb{\zeta}}

\newcommand{\mbDelta}{\mb{\Delta}}
\newcommand{\mbGamma}{\mb{\Gamma}}
\newcommand{\mbLambda}{\mb{\Lambda}}
\newcommand{\mbOmega}{\mb{\Omega}}
\newcommand{\mbPhi}{\mb{\Phi}}
\newcommand{\mbPi}{\mb{\Pi}}
\newcommand{\mbPsi}{\mb{\Psi}}
\newcommand{\mbSigma}{\mb{\Sigma}}
\newcommand{\mbTheta}{\mb{\Theta}}
\newcommand{\mbUpsilon}{\mb{\Upsilon}}
\newcommand{\mbXi}{\mb{\Xi}}

\newcommand{\dif}{\mathop{}\!\mathrm{d}}
\newcommand{\diag}{\textrm{diag}}
\newcommand{\supp}{\textrm{supp}}

\newcommand{\E}{\mathbb{E}}
\newcommand{\Var}{\mathbb{V}\textrm{ar}}

\newcommand{\bbA}{\mathbb{A}}
\newcommand{\bbB}{\mathbb{B}}
\newcommand{\bbC}{\mathbb{C}}
\newcommand{\bbD}{\mathbb{D}}
\newcommand{\bbE}{\mathbb{E}}
\newcommand{\bbF}{\mathbb{F}}
\newcommand{\bbG}{\mathbb{G}}
\newcommand{\bbH}{\mathbb{H}}
\newcommand{\bbI}{\mathbb{I}}
\newcommand{\bbJ}{\mathbb{J}}
\newcommand{\bbK}{\mathbb{K}}
\newcommand{\bbL}{\mathbb{L}}
\newcommand{\bbM}{\mathbb{M}}
\newcommand{\bbN}{\mathbb{N}}
\newcommand{\bbO}{\mathbb{O}}
\newcommand{\bbP}{\mathbb{P}}
\newcommand{\bbQ}{\mathbb{Q}}
\newcommand{\bbR}{\mathbb{R}}
\newcommand{\bbS}{\mathbb{S}}
\newcommand{\bbT}{\mathbb{T}}
\newcommand{\bbU}{\mathbb{U}}
\newcommand{\bbV}{\mathbb{V}}
\newcommand{\bbW}{\mathbb{W}}
\newcommand{\bbX}{\mathbb{X}}
\newcommand{\bbY}{\mathbb{Y}}
\newcommand{\bbZ}{\mathbb{Z}}

\newcommand{\cA}{\mathcal{A}}
\newcommand{\cB}{\mathcal{B}}
\newcommand{\cC}{\mathcal{C}}
\newcommand{\cD}{\mathcal{D}}
\newcommand{\cE}{\mathcal{E}}
\newcommand{\cF}{\mathcal{F}}
\newcommand{\cG}{\mathcal{G}}
\newcommand{\cH}{\mathcal{H}}
\newcommand{\cI}{\mathcal{I}}
\newcommand{\cJ}{\mathcal{J}}
\newcommand{\cK}{\mathcal{K}}
\newcommand{\cL}{\mathcal{L}}
\newcommand{\cM}{\mathcal{M}}
\newcommand{\cN}{\mathcal{N}}
\newcommand{\cO}{\mathcal{O}}
\newcommand{\cP}{\mathcal{P}}
\newcommand{\cQ}{\mathcal{Q}}
\newcommand{\cR}{\mathcal{R}}
\newcommand{\cS}{\mathcal{S}}
\newcommand{\cT}{\mathcal{T}}
\newcommand{\cU}{\mathcal{U}}
\newcommand{\cV}{\mathcal{V}}
\newcommand{\cW}{\mathcal{W}}
\newcommand{\cX}{\mathcal{X}}
\newcommand{\cY}{\mathcal{Y}}
\newcommand{\cZ}{\mathcal{Z}}

\newcommand{\trans}{\mathsf{T}}
\newcommand{\naturals}{\mathbb{N}}
\newcommand{\reals}{\mathbb{R}}

\newcommand{\distNormal}{\mathcal{N}}
\newcommand{\distGamma}{\mathrm{Gamma}}
\newcommand{\distBernoulli}{\mathrm{Bern}}
\newcommand{\distBinomial}{\mathrm{Bin}}
\newcommand{\distCategorical}{\mathrm{Cat}}
\newcommand{\distDirichlet}{\mathrm{Dir}}
\newcommand{\distMultinomial}{\mathrm{Mult}}
\newcommand{\distPolyaGamma}{\mathrm{PG}}
\newcommand{\distMNIW}{\mathrm{MNIW}}
\newcommand{\distPoissonProcess}{\mathrm{PP}}

\newcommand{\dtmax}{\Delta t_{\mathsf{max}}}

\newacronym{KL}{kl}{Kullback-Leibler}
\newacronym{ELBO}{elbo}{\emph{evidence lower bound}}
\newacronym{EM}{em}{\emph{expectation-maximization}}
\newacronym{PPCA}{ppca}{probabilistic principal components analysis}

\newacronym{SVI}{svi}{stochastic variational inference}
\newacronym{GMM}{gmm}{Gaussian mixture model}
\newacronym{HMM}{hmm}{hidden Markov model}
\newacronym{IO-HMM}{io-hmm}{input-output hidden Markov model}
\newacronym{LDS}{lds}{linear dynamical system}
\newacronym{SLDS}{slds}{switching linear dynamical system}
\newacronym{AR-HMM}{ar-hmm}{autoregressive hidden Markov model}


\title{STAT215: Assignment 1}
% \author{Your Name Here}
\date{Due: January 30, 2020 at 11:59pm PT}

\begin{document}

\maketitle

\textbf{Problem 1:}  \textit{The negative binomial distribution.} 

Consider a coin with probability~$p$ of coming up heads.  The number of coin flips before seeing a `tails' follows a geometric distribution with pmf
\begin{align*}
    \Pr(X=k; p) &= p^k \, (1-p).
\end{align*}
The number of coin flips before seeing~$r$ tails follows a \emph{negative binomial} distribution with parameters~$r$ and~$p$.

\begin{enumerate}[label=(\alph*)]
    \item Derive the probability mass function~$\Pr(X=k; r, p)$ of the negative binomial distribution.  Explain your reasoning.
    
    \item The geometric distribution has mean~$p / (1-p)$ and variance~$p / (1-p)^2$.  Compute the mean and variance of the negative binomial distribution.  Plot the variance as a function of the mean for fixed~$p$ and varying~$r$.  How does this compare to the Poisson distribution?
    
    \item Rewrite the negative binomial pmf in terms of the mean~$\mu$ and the dispersion parameter~$r$.  Show that as~$r \to \infty$ with~$\mu$ fixed, the negative binomial converges to a Poisson distribution with mean~$\mu$.
    
    \item The gamma distribution is a continuous distribution on~$(0, \infty)$ with pdf
    \begin{align*}
        p(x; \alpha, \beta) &= \frac{\beta^\alpha}{\Gamma(\alpha)} x^{\alpha -1} e^{-\beta x},
    \end{align*}
    where~$\Gamma(\cdot)$ denotes the gamma function, which has the property that~$\Gamma(n) = (n-1)!$ for positive integers~$n$.  Show that the negative binomial is the marginal distribution over~$X$ where~${X \sim \mathrm{Poisson}(\mu)}$ and~${\mu \sim \mathrm{Gamma}(r, (1-p)/p )}$, integrating over~$\mu$.  In other words, show that the negative binomial is equivalent to an infinite mixture of Poissons with gamma mixing measure. 
    
    \item  Suppose~$X_n \sim \mathrm{NB}(r, p)$ for~$n=1, \ldots, N$ are independent samples of a negative binomial distribution.  Write the log likelihood~$\cL(r, p)$.  Solve for the maximum likelihood estimate (in closed form) of~$\hat{p}$ for fixed~$r$.  Plug this into the log likelihood to obtain the profile likelihood~$\cL(r, \hat{p}(r))$ as a function of~$r$ alone.  

\end{enumerate}

\clearpage


\textbf{Problem 2:}  \textit{The multivariate normal distribution.} 

\begin{enumerate}[label=(\alph*)]

\item In class we introduced a multivariate Gaussian distribution via its representation as a linear transformation~$x = Az + \mu$ where~$z$ is a vector of independent standard normal random variates.  Using the change of variables formula, derive the multivariate Gaussian pdf,
\begin{align*}
    p(x; \mu, \Sigma) &= (2 \pi)^{-D/2} |\Sigma|^{-1/2} \exp \left\{ -\frac{1}{2} (x - \mu)^\trans \Sigma^{-1} (x- \mu) \right\},
\end{align*}
where~$\mu \in \reals^D$ and $\Sigma = AA^\top \in \reals^{D \times D}$ is a positive semi-definite covariance matrix.

\item Let~$r = \|z\|_2 = (\sum_{d=1}^D z_d^2)^{1/2}$ where~$z$ is a vector of standard normal variates, as above.  We will derive its density function. 
\begin{enumerate}[label=(\roman*)]
    \item Start by considering the~$D=2$ dimensional case and note that~$p(r) \, \mathrm{d}r$ equals the probability mass assigned by the multivariate normal distribution to the infinitesimal shell at radius~$r$ from the origin.  
    
    \item Generalize your solution to $D > 2$ dimensions, using the fact that the surface area of the $D$-dimensional ball with radius $r$ is $2r^{D-1} \pi^{D/2} / \Gamma(D/2)$.
    
    \item Plot this density for increasing values of dimension~$D$. What does this tell your about the distribution of high dimensional Gaussian vectors?  
    \item Now use another change of variables to derive the pdf of~$r^2$, the sum of squares of the Gaussian variables. The squared 2-norm follows a $\chi^2$ distribution with $D$ degrees of freedom. Show that it is a special case of the gamma distribution introduced in Problem 1.
    
\end{enumerate}


\item Rewrite the multivariate Gaussian density in natural exponential family form with parameters~$J$ and $h$.  How do its natural parameters relate to its mean parameters~$\mu$ and~$\Sigma$? What are the sufficient statistics of this exponential family distribution?  What is the log normalizer?  Show that the derivatives of the log normalizer yield the expected sufficient statistics. 

\item  Consider a directed graphical model on a collection of scalar random variables $(x_1, \ldots, x_D)$.  Assume that each variable $x_d$ for~$d > 1$ has exactly one parent in the directed graphical model, and let the index of the parent of~$x_d$ be denoted by~$\mathsf{par}_d \in \{1, \ldots, d-1\}$.  The joint distribution is then given by,
\begin{align*}
    x_1 &\sim \cN(0, \beta^{-1}), \\
    x_{d} &\sim \cN(x_{\mathsf{par}_d} + b_d; \beta^{-1}) \qquad \text{ for } d=2, \ldots, D.
\end{align*}
The parameters of the model are~$\beta, \{b_d\}_{d=2}^D$.  Show that the joint distribution is a multivariate Gaussian and find a closed form expression the precision matrix, $J$.  How does the precision matrix change in the two-dimensional model where each~$x_d \in \reals^2$, $\beta^{-1}$ is replaced by $\beta^{-1}I$, and $b_d \in \reals^2$?

\end{enumerate}

\clearpage

\textbf{Problem 3:} \textit{Bayesian linear regression.}  

Consider a regression problem with datapoints~$(x_n, y_n) \in \reals^{D} \times \reals$. We begin with a linear model,
\begin{align*}
    y_n = w^\trans x_n + \epsilon_n;  \quad \epsilon_n \sim \cN(0, \beta^{-1}),
\end{align*}
where~$w \in \reals^{D}$ is a vector of regression weights and~$\beta \in \reals_+$ specifies the precision (inverse variance) of the errors~$\epsilon_n$. 

\begin{enumerate}[label=(\alph*)]

\item Assume a standard normal prior $w_i \sim \cN(0, \alpha^{-1})$.  Compute the marginal likelihood
\begin{align*}
    p(\{x_n, y_n\}_{n=1}^N; \alpha, \beta) &= \int p(w; \alpha) \, p(\{(x_n, y_n)\}_{n=1}^N \mid w; \beta) \, \mathrm{d}w.
\end{align*}

\item Now consider a ``spike-and-slab'' prior distribution on the entries of~$w$.  Let~$z \in \{0, 1\}^{D}$ be a binary vector specifying whether the corresponding entries in~$w$ are nonzero.  That is, if~$z_{i}=0$ then~$w_{i}$ is deterministically zero; otherwise,~$w_{i} \sim \cN(0, \alpha^{-1})$ as above.  We can write this as a degenerate Gaussian prior
\begin{align*}
    p(w \mid z) &= \prod_{i=1}^{D} \cN(w_{i} \mid 0, z_{i} \alpha^{-1}).
\end{align*}
Compute the marginal likelihood~$p(\{(x_n, y_n)\}_{n=1}^N \mid z, \alpha, \beta)$.  How would you find the value of~$z$ that maximizes this likelihood?

\item Suppose that each datapoint has its own precision~$\beta_n$.  Compute the posterior distribution
\begin{align*}
    p(w \mid \{(x_n, y_n, \beta_n)\}_{n=1}^N, \alpha).
\end{align*}
How does the posterior mean compare to the ordinary least squares estimate?

\item Finally, assume the per-datapoint precisions~$\beta_n$ are not directly observed, but are assumed to be independently sampled from a gamma prior distribution,
\begin{align*}
    \beta_n &\sim \mathrm{Gamma}(a, b),
\end{align*}
which has the property that~$\E[\beta_n] = a/b $ and $\E[\ln \beta_n] = \psi(a) - \ln b$ where~$\psi$ is the digamma function.  Then, the errors~$\epsilon_n$ are marginally distributed according to the Student's t distribution, which has heavier tails than the Gaussian and hence is more robust to outliers. 

Compute the conditional distribution $p(\beta_n \mid x_n, y_n, w, a, b)$, and compute the expected log joint 
\begin{align*}
    \cL(w') &= \E_{p(\beta_n \,|\, x_n, y_n, w, a, b)} \left[ \log p(\{(x_n, y_n, \beta_n)\}_{n=1}^N, w'; \alpha, a, b) \right].
\end{align*}
What value of~$w$ maximizes the expected log joint probability?  Describe an EM procedure to search for,
\begin{align*}
    w^* &= \argmax p(w \mid \{(x_n, y_n)\}_{n=1}^N, \alpha, a, b).
\end{align*}

\end{enumerate}

\clearpage

\textbf{Problem 4:} \textit{Multiclass logistic regression applied to larval zebrafish behavior data.}  

Follow the instructions in this Google Colab notebook to implement a multiclass logistic regression model and fit it to larval zebrafish behavior data from a recent paper: 
\url{https://colab.research.google.com/drive/1moN5CYNsyxeOSUOmN-QMyqEZwgLSBsjY}.  Once you're done, save the notebook in \texttt{.ipynb} format, print a copy in \texttt{.pdf} format,
and submit these files along with the rest of your written assignment.


\end{document}
